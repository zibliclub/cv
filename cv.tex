\documentclass[10pt, a4paper]{article}
\usepackage[utf8]{inputenc}
\usepackage[T2A]{fontenc}
\usepackage[russian]{babel}
\usepackage{geometry}
\usepackage{changepage}
\usepackage[shortlabels]{enumitem}
\usepackage{microtype, xspace}
\usepackage{datetime}
\usepackage{hyperref}

\usepackage{import}
\usepackage[mocha,styleAll]{catppuccinPalette}

\newcommand{\hmargin}{2.5cm}
\geometry{top=2.0cm,
bottom=2.0cm,
left=\hmargin,
right=\hmargin}
\hypersetup{colorlinks=true, linkcolor=cyan, urlcolor=cyan}

\newdateformat{monthyeardate}{\monthname[\THEMONTH] \THEYEAR}

\pagenumbering{gobble}
\pagestyle{empty}
\setlength{\parindent}{0pt}
\setlist{topsep=0pt, itemsep=0pt}

% Footer configuration
%--------------------------------------------%
\usepackage{fancyhdr, extramarks}
\pagestyle{fancy}

\fancyhf{}
\renewcommand{\headrulewidth}{0pt}
\fancyfoot[C]{\small \textit{Исходный код доступен тут \textup{\href{https://github.com/zibliclub/cv.git}{\textcolor{CtpBlue}{\texttt{github.com/zibliclub/cv}}}}}}

\setlength{\headheight}{14pt} 
%--------------------------------------------%
% End footer configuration

% BEGIN CUSTOM MACROS
% --------------------------------------------- %
\newcommand{\rangesep}{to\xspace}  % for date ranges
\renewcommand{\date}[1]{\textit{#1}}
\newcommand{\location}[1]{\textit{#1}}
\newcommand{\contentType}[1]{\textit{#1}}

\newcommand{\heading}[1]{
    \makebox[0pt][l]{\Large \sc \hspace{2pt}#1}
    \rule[-0.7ex]{\columnwidth}{0.5pt}\vspace{1.0ex}
    }

    \newcommand{\subheading}[1]{{\bfseries #1}}
    \newcommand{\generalInfoSkip}{\vspace{0.8ex}}
    \newcommand{\subheadSkip}{\vspace{2ex}}
    \newcommand{\contentIndent}{\hspace{1em}}
    % --------------------------------------------- %
    % END CUSTOM MACROS

    % BEGIN CUSTOM ENVIRONMENTS
    % --------------------------------------------- %
    \newenvironment{mysection}[1]
    {\vspace{3.5ex}
    \heading{#1}
    \begin{adjustwidth}{0.3cm}{0cm}}
    {\end{adjustwidth} }

    \newenvironment{entry}
    \par
    {\vspace{0.1ex}}
    {}
    % --------------------------------------------- %
    % END CUSTOM ENVIRONMENTS

\begin{document}

\textbf{\textcolor{CtpMauve}{\LARGE Заблоцкий Данил}}

\vspace{1.5ex}
GitHub: \href{https://github.com/zibliclub}{\textcolor{CtpBlue}{\texttt{www.github.com/zibliclub}}}\\
Behance: \href{https://www.behance.net/zibliclubo9570}{\textcolor{CtpBlue}{\texttt{www.behance.net/zibliclub}}}\\
Dribbble: \href{https://dribbble.com/zibliclub}{\textcolor{CtpBlue}{\texttt{www.dribbble.com/zibliclub}}}\\
Email: \href{mailto:zibliclubofficial@gmail.com}{\textcolor{CtpBlue}{\texttt{zibliclubofficial@gmail.com}}}\\
Telegram: \href{https://t.me/zibliclubofficial}{\textcolor{CtpBlue}{\texttt{@zibliclubofficial}}}\\
Обновлено: 2 ноября 2024 \\

\begin{mysection}{\textcolor{CtpSky}{Основная информация}}
	\textbf{Гражданство:} Россия

	\generalInfoSkip
	\textbf{Место проживания:} Омск

	\generalInfoSkip
	\textbf{Языки:} Русский, Английский

	\generalInfoSkip
	\textbf{Навыки:} ясное изложение мыслей, умение слушать и слышать, четкое и понятное для человека техническое письмо;
	вводное обучение математике, программированию;
	многолетний опыт использования Figma, Photoshop, Illustrator, решение практических задач, проведение UX исследований, применение Google Material Design и iOS гайдлайнов;
	верстка с помощью LaTeX.
\end{mysection}


\begin{mysection}{\textcolor{CtpSky}{Портфолио}}
  \subheading{\textcolor{CtpYellow}{vakansiy.net -- Агрегатор вакансий}} \hfill \contentType{Pet-проект, Design System и Prototype}
	\begin{entry}
    \hspace{1em} \textls[-5]{Текущий \textbf{флагман} среди проектов, в данный момент ведется активная работа. При оценке способностей просьба основываться на данный проект. Важное уточнение, что проект разбит на несколько файлов, особое внимание сейчас уделено} \href{https://www.figma.com/design/ahZ7ZW527l8z3Oc8hZdrcW/vakansiy.net-%7C-Design-System?node-id=0-1&t=pQiIEndErpFgv3o4-1}{\textcolor{CtpBlue}{Design System}}, \textls[-5]{так как Frontend готовит компоненты}.

		\hspace{1em} Посмотреть:
    \href{https://www.figma.com/design/9uufnLLIvYzh0oUL5MfHlB/vakansiy.net?node-id=0-1&t=feB9W8mGXUOVGMzX-1}{\textcolor{CtpBlue}{\texttt{Figma File | vakansiy.net}}}

	\end{entry}

	\subheadSkip
  \subheading{\textcolor{CtpYellow}{ОмГУ -- Оценка остаточных знаний}} \hfill \contentType{Работа в рамках проектной деятельности}
	\begin{entry}
		\hspace{1em} Сервис для студентов и преподавателей, облегчающий тестирование студентов. Данный проект создавался в рамках дисциплины в университете.

		\hspace{1em} Посмотреть:
    \href{https://www.figma.com/design/p7two6gFnQznQaPxpZzoTZ/%D0%9E%D0%BC%D0%93%D0%A3-%7C-%D0%9E%D1%86%D0%B5%D0%BD%D0%BA%D0%B0-%D0%BE%D1%81%D1%82%D0%B0%D1%82%D0%BE%D1%87%D0%BD%D1%8B%D1%85-%D0%B7%D0%BD%D0%B0%D0%BD%D0%B8%D0%B9?node-id=1-12&t=JBw6oQGhnahC4OsK-1}{\textcolor{CtpBlue}{\texttt{Figma File | Омгу -- Оценка остаточных знаний}}}
	\end{entry}

	\subheadSkip
  \subheading{\textcolor{CtpYellow}{ВКурсе (VCourse)}} \hfill \contentType{Pet-проект, UI/UX Design и SwiftUI Dev.}
	\begin{entry}
		\hspace{1em} Мобильное приложение для университета. В данный момент ведется работа на проектом, но большинство времени уделено изучению SwiftUI, так что Figma File -- в \textbf{ужасном} состоянии. В данный CV он добавлен как пример использования гайдлайнов.

		\hspace{1em} Посмотреть:
    \href{https://www.figma.com/design/Dsh5DtdvobsrDORQOM6Rlj/UNIK-%7C-iOS-App-%7C-Old?node-id=1-10&t=mQTpEo2yxfbbsfZY-1}{\textcolor{CtpBlue}{\texttt{Figma File | VCourse}}}
	\end{entry}
\end{mysection}

\begin{mysection}{\textcolor{CtpSky}{Образование}}
	\subheading{ФГАОУ ВО «ОмГУ им. Ф.М. Достоевского», Факультет Цифровых Технологий и Кибербезопасности} \hfill \location{Омск, Россия}

	\textit{Бакалавр по Прикладной Математике} \hfill \date{Сентябрь 2022 --- настоящее время}
	\end{entry}

	\subheadSkip
	\subheading{«БОУ г. Омска Инженерно-технологический лицей №25»} \hfill \location{Омск, Россия}

	\textit{Среднее общее образование} \hfill \date{Сентябрь 2011 --- Июнь 2022}
\end{mysection}

\end{document}
